\documentclass[10pt,a4paper]{article}
\usepackage[utf8]{inputenc}
\usepackage{amsmath}
\usepackage{amsfonts}
\usepackage{amssymb}
\usepackage{amsthm}
\usepackage{graphicx}
\usepackage[margin=0.6in]{geometry}
\setcounter{section}{1}
\begin{document}



\section*{\textit{Prüfungsrelevante} Verfahren, Sätze und Rechenregeln}
\section{Diskrete Strukturen}
\subsection{Mengenlehre und Kombinatorik}
\begin{itemize}
\item zwei Mengen $A$ und $B$ sind gleich wenn sie die selben Elemente haben, d.h. wenn $A \subseteq B \land B \subseteq A$
\item Beachte z.B. dass $\left\lbrace\left\lbrace 1,2\right\rbrace, 7\right\rbrace \nsubseteq \mathbb{N}$
\item Schnitt und Vereinigung sind kommutativ, assoziativ, distributiv in beide Richtungen; für Beweise kann es nützlich sein sich die Definitionen dieser Operationen in Erinnerung zu rufen; $\overline{A\cup B}=\overline{A}\cap \overline{B}, \overline{A\cap B}=\overline{A}\cup \overline{B}$
\item $A\times B=\left\lbrace (a,b)\mid a\in A \land b\in B\right\rbrace$ heißt \textbf{kartesisches Produkt} oder \textbf{Produktmenge};  $\vert A\times B\vert =\vert A\vert \cdot \vert B\vert$
\item \textbf{Potenzmenge} $\mathcal{P}(A)$ ist die Menge aller (auch unechten) Teilmengen von $A$, $\vert \mathcal{P}(A)\vert =2^{\vert A\vert}$, es gilt stets $\emptyset \in \mathcal{P}(A)$
\item $\begin{pmatrix} n\\ k\end{pmatrix}=\dfrac{n!}{k!(n-k)!}$
\item \textbf{Handschlaglemma:} Anzahl der Teilnehmer einer Konferenz, die einer ungeraden Anzahl von Teilnehmern die Hand geben, ist immer gerade

%genauer siehe Graphentheorie, ich hab versucht da eine Form zu finden die man sich besser merken kann aber das scheint tatsächlich nur in der Graphentheorie eine größere Rolle zu spielen; daher zeile evtl. noch zu streichen 

\end{itemize}



\subsection{Abbildungen}
\begin{itemize}
\item für $f: A\rightarrow B, A'\subseteq A$ heißt $f[A']=\left\lbrace f(a)\mid a \in A'\right\rbrace$ Bild von $A'$ unter $f$
\item \textbf{injektiv:} $f(a_{1})=f(a_{2}) \Rightarrow a_{1}=a_{2}$ ("für jedes $b\in B$ existiert höchstens ein $a\in A$ mit $f(a)=b$")\\
Beweise über Gegenbeispiel oder $f(a_{1})=f(a_{2})$ setzen 
\item \textbf{surjektiv:} $f[A]=B$ ("für jedes $b\in B$ existiert mindestens ein $a\in A$ mit $f(a)=b$")\\ Beweise über Gegenbeispiel oder Definitionsbereich der Umkehrfunktion untersuchen 
\item \textbf{bijektiv:} injektiv und surjektiv ("für jedes $b\in B $ existiert genau ein $a\in A$ mit $f(a)=B$")
\item für $f$ \textit{injektiv (!!)}  definieren wir $\boldsymbol{f^{-1}: }f[A]\rightarrow A, b\mapsto f^{-1}(b)=a$ mit $f^{-1}(b)=a$  g.d.w. $f(a)=b$
\item für $f:A\rightarrow B, g: B\rightarrow C$ ist \textbf{Komposition} $ g\circ f:A\rightarrow C, x \mapsto g(f(x))$ ($\Rightarrow$ von rechts nach links ausführen!!)
\end{itemize}



\subsection{Permutationen}
\begin{itemize}
\item \textbf{Permutation} von $X$ ist bijektive Abbildung von $X$ nach $X$, für $X=\left\lbrace 1,\dotsc, n\right\rbrace$ ist $S_{n}$ Menge aller Permutationen und $\pi \in S_{n}$ mit $\pi=\begin{pmatrix}1&\dotsc& n\\ \pi(1)&\dotsc &\pi(n)\end{pmatrix} $ und $\vert S_{n}\vert =n!$
\item \textbf{k-Zyklus} = k-Tupel der Form $(a_{1},\dotsc,a_{k})$ mit $\pi(a_{k})=a_{1}, \pi(a_{i})=a_{i+1}$, jedes Element von $S_{n}$ kann als Komposition elementfremder Zyklen notiert werden: $\begin{pmatrix}1&2& 3&4&5\\ 2&3&1&5&4\end{pmatrix}=(1,2,3)\circ (4,5)=(123)(45)$
\item bei elementfremden Zyklen ist Reihenfolge egal: $(123)(45)=(45)(123)$; Elemente die auf sich selbst abgebildet werden heißen \textbf{Fixpunkte} und müssen nicht notiert werden: $(123)(4)=(123)$; mit welchem Element im Zyklus angefangen wird ist Egal: $(123)(45)=(312)(54)$ 
\item \textbf{Transposition} = 2-Zyklus, jedes Element von $S_{n}$ kann mit Transpositionen geschrieben werden als:\\ $(a_{1},\dotsc, a_{k})=(a_{1}a_{2})(a_{2}a_{3})\dotsc(a_{k-1}a_{k})$ (nicht elementfremd $\Rightarrow$ Reihenfolge wichtig!!)
\item bei Komposition von Permutationen für jede Zahl von rechts nach links durchgehen: $\underbrace{(123)}_{\text{\textit{(2)}}}\underbrace{(35)}_{\text{\textit{(1)}}}=\underbrace{(1235)}_{\text{\textit{(3)}}}$\\5 wird in \textit{(1)} auf 3 abgebildet, in \textit{(2)} wird 3 auf 1 abgebildet, also $5\rightarrow 3 \rightarrow 1$ und damit $5\rightarrow 1$  in \textit{(3)} 
\end{itemize}



\subsection{Beweis mittels vollständiger Induktion (Beispiel)}
\textit{Beweis.}
Die Aussage $A_{n}$ sei $\sum_{k=0}^{n} q^{k}= \dfrac{1-q^{n+1}}{1-q}$ mit $n \in \mathbb{N}, q \in \mathbb{R}, q \neq 1$.\\\\
(IA): $n_{0}=0$: $\sum_{k=0}^{0} q^{k}=q^{0}=1= \dfrac{1-q^{0+1}}{1-q}$ \hspace{0.3cm} w.A.\\
\hspace*{1cm}$\Rightarrow$ Es gilt $A_{0}$\\\\
(IV): $\forall \tilde{n}: n_{0} \leq \tilde{n} \leq n:\sum_{k=0}^{\tilde{n}} q^{k}= \dfrac{1-q^{\tilde{n}+1}}{1-q}$\\\\
(IS):$\sum_{k=0}^{n+1} q^{k}= \sum_{k=0}^{n} q^{k}+q^{n+1}\stackrel{\text{(IV)}}{=}\dfrac{1-q^{n+1}}{1-q}+q^{n+1}=\dfrac{1-q^{n+1}+(1-q)q^{n+1}}{1-q}$\\
\hspace*{2.23cm}$=\dfrac{1-q^{n+1}+q^{n+1}-q^{n+2}}{1-q}=\dfrac{1-q^{(n+1)+1}}{1-q}$\\\\
\hspace*{1cm}$\Rightarrow$ Damit ist die Behauptung für alle $n \in \mathbb{N}$ vollständig bewiesen
\qed\\\\
\begin{itemize}
\item "Die Aussage $A_{n}$ sei..." nur in VL und AuD Skript, \textit{evtl.} wird sonst aber z.Z.: erwartet; IV muss auch nicht unbedingt notiert werden
\item alles nochmal mit (n+1)+1 hinschreiben ist nicht nötig
\item Varianten: $A_{n} \Rightarrow A_{n+1}$/ aus $A_{n}$ folgt $A_{n+1}$ für alle $n\in \mathbb{N}$\\
 w.A. /Folglich gilt $A_{n}$ für alle $n \in \mathbb{N}, n \geq n_{0}$
\item Beachte dass oft auch nur für $n \in \mathbb{N}, n\geq k$ bewiesen wird (kein $\tilde{n}$)!! und $n_{0}=0$ nicht immer gelten muss
\end{itemize}



\subsection{Zahlentheorie}
\begin{itemize}
\item $n\in \mathbb{N}, n\geq 1$ kann \textit{eindeutig} geschrieben werden als $n=\prod_{i=1}^{k} p_{i}^{\alpha_{i}}$ mit $p_{i}$ prim, $\alpha_{i} \in \mathbb{N}\Rightarrow\#\text{Teiler}=\prod_{i} (\alpha_{i}+1)$
\item für $a,b \in \mathbb{N}$ gilt $a\mid b \Leftrightarrow \exists k :k\in\mathbb{N}\land  ak=b;\;\;\; a\mid b_{1}\land a\mid b_{2} \Rightarrow a\mid (b_{1}+b_{2}) \land a\mid (b_{1}-b_{2})$
\item für $m,n \in \mathbb{Z}$ mit $n>0$ gilt $\exists q,r:( q,r \in \mathbb{Z}\land m=nq+r\land 0\leq r<n)$\\ $m \text{ mod } n:=r,\;\;$
für $a \text{ mod } n =b \text{ mod } n$ schreibe $a \equiv b \text{ mod } n$
\item \textbf{Homomorphieregel:} $(a \text{ mod } n+b \text{ mod } n) \text{ mod } n=(a+b) \text{ mod } n\;\;\;\;\;$ (analog für $\cdot\;$)
\item $\text{ggT}(0,0)=0;\;\;\;\text{kgV}(m,0)=\text{kgV}(0,n)=0;\;\;\;\text{ggT}(m,n)\cdot \text{kgV}(m,n)=m\cdot n;\;\;\; m,n$ teilerfremd für ggT$(m,n)=1$
\item \textbf{Euklidischer Algorithmus} (EA, keine offizielle Abkürzung):
\item \textbf{Erweiterter Euklidischer Algorithmus} (EEA, keine offizielle Abkürzung): 
\item \textbf{Lemma von Bézout}
\item \textbf{Al Kashi's Trick}
\item \textbf{chinesischer Restsatz}

\end{itemize}



\subsection{Gruppentheorie}



\subsection{Kryptographie}
\begin{itemize}
\item \textbf{diskreter Logarithmus:} 
\item \textbf{Diffie-Hellman-Merkle:}
\item \textbf{RSA:}
\begin{enumerate}
\item Bob wählt zufällig 2 Primzahlen $p,q$ und berechnet $n:=pq$
\item Bob wählt zufällig $d\in \mathbb{Z}_{\phi(n)}^{*}$ und berechnet $i,h\in \mathbb{Z}$ mit $ i\cdot d+h\cdot \phi(n)=\text{ggT}(d,\phi(n))=1$ (EEA)

%prüfen ob dieser Punk tatsächlich korrekt 

\item $n$ und $i$ sind öffentliche Schlüssel und werden an Alice weitergegeben, $d$ ist privater Schlüssel
\item Alice schickt $c=m^{i} \text{ mod } n$ an Bob mit Nachricht $m$ ($0\leq m <n$) 
\item Bob berechnet $m=c^{d} \text{ mod } n$ (Potenzieren mit Al Kashi oder Euler-Fermat oder Zahl zerlegen)
\end{enumerate}
\end{itemize}


\subsection{Ungerichtete Graphen}



\subsection{Gerichtete Graphen}



\subsection{Aussagenlogik}



\subsection{Relationen}
%beachte hier evtl auch Grundlagen aus VL 1 und VL3
\end{document}