%Grundsätzliches: Enthalten sind die Gebieten die Frau Henke schon als Schwerpunkte angekündigt hat (und natürliche das nicht ausdrücklich geforderte Grundwissen auf dem sie aufbauen).
%Das meiste ist eher knapp gehalten, extrem grundlegende Dinge sind nicht nochmal erklärt weil ich denke dass das beim lernen eher stört.
%Selbstverständlich kannst du auch noch kürzen/ergänzen 

\documentclass[10pt,a4paper]{article}
\usepackage[utf8]{inputenc}
\usepackage{amsmath}
\usepackage{amsfonts}
\usepackage{amssymb}
\usepackage{graphicx}
\usepackage[margin=0.6in]{geometry}
\setcounter{section}{1}
\begin{document}



\section*{\textit{Prüfungsrelevante} Verfahren, Sätze und Rechenregeln}
\section{Diskrete Strukturen}



\subsection{Mengenlehre}



\subsection{Abbildungen}



\subsection{Beweis mittels vollständiger Induktion}



\subsection{Zahlentheorie}



\subsection{Gruppentheorie}



\subsection{Graphentheorie}




\subsection{Aussagenlogik}



\subsection{Relationen}
\end{document}