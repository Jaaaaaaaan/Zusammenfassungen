\documentclass[10pt,a4paper]{article}
\usepackage[utf8]{inputenc}
\usepackage{amsmath}
\usepackage{amsfonts}
\usepackage{amssymb}
\usepackage{amsthm}
\usepackage{graphicx}
\usepackage[margin=0.6in]{geometry}
\setcounter{section}{1}
\begin{document}



\section*{\textit{Prüfungsrelevante} Verfahren, Sätze und Rechenregeln}
\section{Diskrete Strukturen}
\subsection{Mengenlehre und Kombinatorik}
\begin{itemize}
\item zwei Mengen $A$ und $B$ sind gleich wenn sie die selben Elemente haben, d.h. wenn $A \subseteq B \land B \subseteq A$
\item Beachte z.B. dass $\left\lbrace\left\lbrace 1,2\right\rbrace, 7\right\rbrace \nsubseteq \mathbb{N}$
\item Schnitt und Vereinigung sind kommutativ, assoziativ, distributiv in beide Richtungen; für Beweise kann es nützlich sein sich die Definitionen dieser Operationen in Erinnerung zu rufen; $\overline{A\cup B}=\overline{A}\cap \overline{B}, \overline{A\cap B}=\overline{A}\cup \overline{B}$
\item $A\times B=\left\lbrace (a,b)\mid a\in A \land b\in B\right\rbrace$ heißt \textbf{kartesisches Produkt} oder \textbf{Produktmenge};  $\vert A\times B\vert =\vert A\vert \cdot \vert B\vert$
\item \textbf{Potenzmenge} $\mathcal{P}(A)$ ist die Menge aller (auch unechten) Teilmengen von $A$, $\vert \mathcal{P}(A)\vert =2^{\vert A\vert}$, es gilt stets $\emptyset \in \mathcal{P}(A)$
\item $\begin{pmatrix} n\\ k\end{pmatrix}=\dfrac{n!}{k!(n-k)!}$
\item \textbf{Handschlaglemma:} Anzahl der Teilnehmer einer Konferenz, die einer ungeraden Anzahl von Teilnehmern die Hand geben, ist immer gerade

%genauer siehe Graphentheorie, ich hab versucht da eine Form zu finden die man sich besser merken kann aber das scheint tatsächlich nur in der Graphentheorie eine größere Rolle zu spielen; daher zeile evtl. noch zu streichen 

\end{itemize}



\subsection{Abbildungen}
\begin{itemize}
\item für $f: A\rightarrow B, A'\subseteq A$ heißt $f[A']=\left\lbrace f(a)\mid a \in A'\right\rbrace$ Bild von $A'$ unter $f$
\item \textbf{injektiv:} $f(a_{1})=f(a_{2}) \Rightarrow a_{1}=a_{2}$ ("für jedes $b\in B$ existiert höchstens ein $a\in A$ mit $f(a)=b$")\\
Beweise über Gegenbeispiel oder $f(a_{1})=f(a_{2})$ setzen 
\item \textbf{surjektiv:} $f[A]=B$ ("für jedes $b\in B$ existiert mindestens ein $a\in A$ mit $f(a)=b$")\\ Beweise über Gegenbeispiel oder Definitionsbereich der Umkehrfunktion untersuchen 
\item \textbf{bijektiv:} injektiv und surjektiv ("für jedes $b\in B $ existiert genau ein $a\in A$ mit $f(a)=B$")
\item für $f$ \textit{injektiv (!!)}  definieren wir $\boldsymbol{f^{-1}: }f[A]\rightarrow A, b\mapsto f^{-1}(b)=a$ mit $f^{-1}(b)=a$  g.d.w. $f(a)=b$
\item für $f:A\rightarrow B, g: B\rightarrow C$ ist \textbf{Komposition} $ g\circ f:A\rightarrow C, x \mapsto g(f(x))$ ($\Rightarrow$ von rechts nach links ausführen!!)
\end{itemize}



\subsection{Permutationen}
\begin{itemize}
\item \textbf{Permutation} von $X$ ist bijektive Abbildung von $X$ nach $X$, für $X=\left\lbrace 1,\dotsc, n\right\rbrace$ ist $S_{n}$ Menge aller Permutationen und $\pi \in S_{n}$ mit $\pi=\begin{pmatrix}1&\dotsc& n\\ \pi(1)&\dotsc &\pi(n)\end{pmatrix} $ und $\vert S_{n}\vert =n!$
\item \textbf{k-Zyklus} = k-Tupel der Form $(a_{1},\dotsc,a_{k})$ mit $\pi(a_{k})=a_{1}, \pi(a_{i})=a_{i+1}$, jedes Element von $S_{n}$ kann als Komposition elementfremder Zyklen notiert werden: $\begin{pmatrix}1&2& 3&4&5\\ 2&3&1&5&4\end{pmatrix}=(1,2,3)\circ (4,5)=(123)(45)$
\item bei elementfremden Zyklen ist Reihenfolge egal: $(123)(45)=(45)(123)$; Elemente die auf sich selbst abgebildet werden heißen \textbf{Fixpunkte} und müssen nicht notiert werden: $(123)(4)=(123)$; mit welchem Element im Zyklus angefangen wird ist Egal: $(123)(45)=(312)(54)$ 
\item \textbf{Transposition} = 2-Zyklus, jedes Element von $S_{n}$ kann mit Transpositionen geschrieben werden als:\\ $(a_{1},\dotsc, a_{k})=(a_{1}a_{2})(a_{2}a_{3})\dotsc(a_{k-1}a_{k})$ (nicht elementfremd $\Rightarrow$ Reihenfolge wichtig!!)
\item bei Komposition von Permutationen für jede Zahl von rechts nach links durchgehen: $\underbrace{(123)}_{\text{\textit{(2)}}}\underbrace{(35)}_{\text{\textit{(1)}}}=\underbrace{(1235)}_{\text{\textit{(3)}}}$\\5 wird in \textit{(1)} auf 3 abgebildet, in \textit{(2)} wird 3 auf 1 abgebildet, also $5\rightarrow 3 \rightarrow 1$ und damit $5\rightarrow 1$  in \textit{(3)} 
\end{itemize}



\subsection{Beweis mittels vollständiger Induktion (Beispiel)}
\textit{Beweis.}
Die Aussage $A_{n}$ sei $\sum_{k=0}^{n} q^{k}= \dfrac{1-q^{n+1}}{1-q}$ mit $n \in \mathbb{N}, q \in \mathbb{R}, q \neq 1$.\\\\
(IA): $n_{0}=0$: $\sum_{k=0}^{0} q^{k}=q^{0}=1= \dfrac{1-q^{0+1}}{1-q}$ \hspace{0.3cm} w.A.\\
\hspace*{1cm}$\Rightarrow$ Es gilt $A_{0}$\\\\
(IV): $\forall \tilde{n}: n_{0} \leq \tilde{n} \leq n:\sum_{k=0}^{\tilde{n}} q^{k}= \dfrac{1-q^{\tilde{n}+1}}{1-q}$\\\\
(IS):$\sum_{k=0}^{n+1} q^{k}= \sum_{k=0}^{n} q^{k}+q^{n+1}\stackrel{\text{(IV)}}{=}\dfrac{1-q^{n+1}}{1-q}+q^{n+1}=\dfrac{1-q^{n+1}+(1-q)q^{n+1}}{1-q}$\\
\hspace*{2.23cm}$=\dfrac{1-q^{n+1}+q^{n+1}-q^{n+2}}{1-q}=\dfrac{1-q^{(n+1)+1}}{1-q}$\\\\
\hspace*{1cm}$\Rightarrow$ Damit ist die Behauptung für alle $n \in \mathbb{N}$ vollständig bewiesen
\qed\\\\
\begin{itemize}
\item "Die Aussage $A_{n}$ sei..." nur in VL und AuD Skript, \textit{evtl.} wird sonst aber z.Z.: erwartet; IV muss auch nicht unbedingt notiert werden
\item alles nochmal mit (n+1)+1 hinschreiben ist nicht nötig
\item Varianten: $A_{n} \Rightarrow A_{n+1}$/ aus $A_{n}$ folgt $A_{n+1}$ für alle $n\in \mathbb{N}$\\
 w.A. /Folglich gilt $A_{n}$ für alle $n \in \mathbb{N}, n \geq n_{0}$
\item Beachte dass oft auch nur für $n \in \mathbb{N}, n\geq k$ bewiesen wird (kein $\tilde{n}$)!! und $n_{0}=0$ nicht immer gelten muss
\end{itemize}



\subsection{Zahlentheorie}
\begin{itemize}
\item $n\in \mathbb{N}, n\geq 1$ kann \textit{eindeutig} geschrieben werden als $n=\prod_{i=1}^{k} p_{i}^{\alpha_{i}}$ ($p_{i}$ prim, $\alpha_{i} \in \mathbb{N},$ "PFZ")\\ $\Rightarrow\#\text{Teiler von }n=\prod_{i} (\alpha_{i}+1)$
\item für $a,b \in \mathbb{N}$ gilt $a\mid b \Leftrightarrow \exists k :k\in\mathbb{N}\land  ak=b;\;\;\; a\mid b_{1}\land a\mid b_{2} \Rightarrow a\mid (b_{1}+b_{2}) \land a\mid (b_{1}-b_{2})$
\item für $m,n \in \mathbb{Z}$ mit $n>0$ gilt $\exists q,r:( q,r \in \mathbb{Z}\land m=nq+r\land 0\leq r<n)$\\ $m \text{ mod } n:=r,\;\;$
für $a \text{ mod } n =b \text{ mod } n$ schreibe $a \equiv b \text{ mod } n$
\item \textbf{Homomorphieregel:} $(a \text{ mod } n+b \text{ mod } n) \text{ mod } n=(a+b) \text{ mod } n\;\;\;\;\;$ (analog für $\cdot\;$)
\item $\text{kgV}(m,0)=\text{kgV}(0,n)=0;\;\text{ggT}(m,n)\cdot \text{kgV}(m,n)=m\cdot n\;(\Rightarrow$ kgV mit Euklid berechenbar) 
\item \textbf{Euklidischer Algorithmus:}
immer weiter ggT$(m,n)=\text{ggT}(n \text{ mod } m,m)$ berechnen; $\;\text{ggT}(m,n)=m$ falls $m\mid n;\;\; \text{ggT}(0,n)=n;\;\; m,n \text{ teilerfremd }  \Leftrightarrow \text{ggT}(m,n)=1$
\item \textbf{Lemma von Bézout:} $m,n \in \mathbb{N} \Rightarrow \exists a,b :a,b\in \mathbb{Z}\land \text{ggT}(m,n)=am+bn$
\item \textbf{Erweiterter Euklidischer Algorithmus} am Beispiel ("EEA", keine offizielle Abkürzung): \\
\end{itemize}
\begin{tabular}{ c | c c | c }
   & $\underbrace{1008}_{m}$ & $\underbrace{499}_{n}$ & $-q_{i}$ \\ 
 \hline
 1008 & 1 & 0 & \\  
 499 & 0 & 1 &  \\  
$1008 \text{ mod } 499 =10$ & 1 & $0+1\cdot(-q_{i})=-2$ & $1008=499\cdot 2+10\Rightarrow -q_{i}=-2$\\
$499 \text{ mod } 10 =9$ & -49 & $1+(-2)(-q_{i})=99$ & -49  \\
$10 \text{ mod } 9 =1$ & 50 & $-2+99(-q_{i})=-101$ & -1 \\
$9 \text{ mod } 1 =0$&&&
\end{tabular} 
\hspace{0.1cm} $\Rightarrow$\hspace{0.2cm}
\begin{tabular}{ c | c c | c }
   & 1008 & 499 & $-q_{i}$ \\ 
 \hline
 1008 & 1 & 0 & \\  
 499 & 0 & 1 &  \\  
 10 & 1 & -2 & -2\\
 9 & -49 & 99 & -49  \\
 1 & 50 & -101 & -1 \\
\end{tabular} \\\\
in $m$ Spalte wird analog zu $n$ Spalte gerechnet, das ganze bis in der linken Spalte 0 stehen würde \\\\
$\Rightarrow \text{ggT}(1008,499)=1=50\cdot 1008 -101\cdot 499\;\;\;\;\;$ (Bézout Koeffizienten $a=50, b=-101$)
\begin{itemize}
\item \textbf{chinesischer Restsatz:} Seien $0<n_{1},\dotsc ,n_{k}\in \mathbb{N}$ teilerfremd und seien $a_{1},\dotsc ,a_{k} \in \mathbb{Z}$. Dann existiert genau ein $x\in \left\lbrace 0,1,\dotsc,\prod_{i=1}^{k} n_{i}-1\right\rbrace$ mit $x\equiv a_{i} \text{ mod } n_{i}$ für alle $i=1,\dotsc, k$
\item für $k=2:$ Seien $0<m,n\in \mathbb{N}$ teilerfremd und seien $a_{1},a_{2} \in \mathbb{N}$. Dann existiert genau ein $x\in \left\lbrace 0,1,\dotsc,mn-1\right\rbrace$ mit $x\equiv a_{1} \text{ mod } m \land x\equiv a_{2} \text{ mod } n;\;$ anschaulich heißt das, dass ein $m\times n$ Spielbrett eindeutig wie in VL durchnummeriert werden kann wenn $\text{ggT}(m,n)=1$

%warum $a_{1},a_{2} \in \mathbb{N}$, nicht $\mathbb{Z}$??

\end{itemize}




\subsection{Gruppentheorie}
\begin{itemize}
\item Gruppe $(G,\circ)$ (auch $(G;\circ,^{-1},e)$; dann Definition einfach anders formulieren) besteht aus Menge $G$ und innerer Verknüpfung $\circ:G\times G\rightarrow G$ so dass: $\circ$ assoziativ, es existiert  \textbf{neutrales Element} $e$ ($a\circ e=a=e\circ a$ für alle $a\in G$), es existiert \textbf{Inverses} $a^{-1}$ zu jedem $a\in G$ ($a\circ a^{-1}=e=a^{-1}\circ a$)
\item Gruppe heißt \textbf{abelsch}/kommutativ, falls $\circ$ kommutativ ist
\item $\mathbb{Z}_{n}=\left\lbrace 0,\dotsc,n-1\right\rbrace $ bildet mit Addition mod $n$ eine Gruppe; Symmetrien eines Quadrates ("$D_4"/"D_{8}"$)/Dreiecks etc. bilden ebenfalls eine Gruppe (Komposition führt zu Drehungen, Spiegelungen und Identitätsabbildung)
\item \textbf{Nullteiler} mod $n$ sind $a\in \mathbb{Z}_{n}\setminus \left\lbrace 0\right\rbrace$ für die $b\in \mathbb{Z}_{n}\setminus \left\lbrace 0\right\rbrace$ existiert mit $a\cdot b\equiv 0 \text{ mod } n$\\
\textbf{Einheiten} mod $n$ sind $a\in \mathbb{Z}_{n}$ für die $b\in \mathbb{Z}_{n}$ existiert mit $a\cdot b\equiv 1 \text{ mod } n$; 1 ist immer eine Einheit\\
m ist Einheit mod $n\Leftrightarrow$ m ist kein Nullteiler mod $n\Leftrightarrow$ ggT$(m,n)=1$
\item Die Menge der Einheiten mod $n$ heißt $\mathbb{Z}_{n}^{*}$ und bildet eine Gruppe mit Multiplikation mod $n$; es gilt mit PFZ dass $\phi(n):=\vert \mathbb{Z}_{n}^{*}\vert=\prod_{i=1}^{k}(p_{i}^{\alpha_{i}}-p_{i}^{\alpha_{i}-1})=\# \text{ zu } n \text{ teilerfremde Zahlen}$
\item multiplikative Gruppe ist \textbf{zyklisch} falls $g\in G$ existiert mit $G=\left\lbrace g^{j}\mid j \in \mathbb{Z}\right\rbrace$ (Potenzrechengesetze ähnlich wie in $\mathbb{N}$, für additive Gruppen schreibe $G=\left\lbrace jg\mid j \in \mathbb{Z}\right\rbrace$); Erzeuger sind $g\in G,\vert G\vert =n$ so dass $G=\left\lbrace g^{j}\mid j \in \mathbb{Z}_{n}\right\rbrace;$\\
Erzeuger von $\mathbb{Z}_{n}$ sind genau die Elemente von $\mathbb{Z}_{n}^{*}\Rightarrow$ \# Erzeuger$=\phi(n)$ 
\item für $p$ prim ist $(\mathbb{Z}_{p}^{*}, \cdot \text{ mod } n$) zyklisch; \# Erzeuger$=\phi(p-1)$; diese Erzeuger heißen \textbf{Primitivwurzeln}
\item \textbf{Isomorph}=Strukturgleich ("man kann Elemente einfach umbenennen"), jede zyklische Gruppe ist isomorph entweder zu $(\mathbb{Z},+)$ oder einem $(\mathbb{Z}_{n},+ \text{ mod } n)$; als Beweis das zwei Gruppen nicht isomorph sind genügt z.B. "$G_{1}$ ist zyklisch, $G_{2}$ nicht "
\item $U\subseteq G$ heißt \textbf{Untergruppe} von $G$ falls $e\in U;\;\; a\circ b \in U \text { für alle } a,b\in U;\;\; a^{-1} \in U \text{ für alle } a \in U$ 
\item $\langle g \rangle:=\left\lbrace g^{i}\mid i \in \mathbb{Z}\right\rbrace$ ist \textbf{von $\boldsymbol{g}$ erzeugte Untergruppe}
\item $g \circ U=\left\lbrace g\circ u \mid u \in U \right\rbrace$ ist eine \textbf{(Links-)Nebenklasse} ("LNK") von $U,\; \vert g\circ U\vert =\vert U\vert$, 2 LNK sind entweder gleich oder disjunkt $\Rightarrow$ jedes $g\in G$ liegt in genau einer LNK (nämlich $g\circ U$)
\item \textbf{Satz von Lagrange:} $\vert G \vert = \vert G : U \vert \cdot \vert U \vert\;\;\;\;$  ( $\vert G : U \vert=\#$ LNK von $U$ in $G$=Index von $U$ in $G$)\\ $\vert G \vert$ heißt Ordnung von $G$, o$(g):=\vert \langle g\rangle \vert$ Ordnung von $g$
\item \textbf{Satz von Euler-Fermat:} Seien $n\in \mathbb{N}, a\in \mathbb{Z}$ mit ggT$(a,n)=1$. Dann gilt $a^{\phi(n)}\equiv 1 \text{ mod } n$. (bei kleinem Fermat gilt $n$ prim)
\item für Beweise bieten sich oft Verknüpfungstafeln an, folgende Methoden helfen beim effizient rechnen:
\end{itemize}



\subsection{Effizient potenzieren mod $\boldsymbol{n}$}
\begin{itemize}

% al kashi würde ich nicht mitnehmen weil sehr umständlich und nicht ausdrücklich gefordert und wenn euler fermat nicht geht ist man mit zerlegen doch am ende genau so schnell

\item andere Form von Euler-Fermat: $a^{m} \equiv a^{m \text { mod } \phi (n)} \text{ mod n}$, gilt aber auch nur für ggT$(a,n)=1$ !!
\item einfach die Zahlen Stück für Stück mit mod zerlegen
\end{itemize}



\subsection{Kryptographie}
\begin{itemize}
\item für $p$ prim und $g$ Primitivwurzel von $\mathbb{Z}_{p}^{*}$ ist der \textbf{diskrete Logarithmus} von $x \in \mathbb{Z}_{p}^{*}$ zur Basis $g$ die Zahl $m\in \left\lbrace 0,\dotsc,p-2\right\rbrace$ mit  $g^{m}\equiv x \text{ mod } p\;\;\; (m=\text{log}_{g}(x))$; $m$ kann nicht effizient berechnet werden, $x$ aus $g^{m}$ schon
\item \textbf{Diffie-Hellman-Merkle:}
\begin{enumerate}
\item Alice und Bob einigen sich auf Primzahl $p$ und Primitivwurzel $g$ von $\mathbb{Z}_{p}^{*}$
\item Alice wählt geheime Zufallszahl $a$ und berechnet $a'=g^{a} \text{ mod } p$; Bob analog: $b'=g^{b} \text{ mod } p$
\item beide teilen sich $a'$ und $b'$ mit und berechnen das Geheimnis $c=g^{ab} \text{ mod } p=(a')^{b} \text{ mod } p=(b')^{a} \text{ mod } p$
\end{enumerate}
Um damit Nachricht $m\leq c$ zu verschlüsseln:
\begin{enumerate}
\item  schreibe $m$ und $c$ binär als $m=m_{1}\dotsc m_{l},c=c_{1}\dotsc c_{k}$
\item Alice verschickt $v_{1}=m_{1}+c_{1} \text{ mod } 2,\dotsc,v_{l}=m_{l}+c_{l} \text{ mod } 2$; Bob berechnet $m_{i}=v_{i}+c_{i} \text{ mod } 2$
\end{enumerate}
\item \textbf{RSA:}
\begin{enumerate}
\item Bob wählt zufällig 2 Primzahlen $p,q$ und berechnet $n:=pq$
\item Bob wählt zufällig $d\in \mathbb{Z}_{\phi(n)}^{*}$ und berechnet $i,h\in \mathbb{Z}$ mit $ i\cdot d+h\cdot \phi(n)=\text{ggT}(d,\phi(n))=1$ (EEA)

%prüfen ob dieser Punk tatsächlich korrekt 

\item $n$ und $i$ sind öffentliche Schlüssel und werden an Alice weitergegeben, $d$ ist privater Schlüssel
\item Alice schickt $c=m^{i} \text{ mod } n$ an Bob mit Nachricht $m$ ($0\leq m <n$) 
\item Bob berechnet $m=c^{d} \text{ mod } n$ 
\end{enumerate}
\end{itemize}


\subsection{Ungerichtete Graphen}



\subsection{Gerichtete Graphen}



\subsection{Aussagenlogik}
\begin{itemize}
\item $\top/\perp=$ wahr/falsch; $\land, \lor$ kommutativ, assoziativ, distributiv in beide Richtungen\\ \textbf{De Morgansche Gesetze:} $\lnot (x\land y)=\lnot x \lor \lnot y, \lnot (x\lor y)=\lnot x \land \lnot y$
\item jeder \textbf{Ausdruck} $A$ (=Verbindung von $\top,\perp,\lnot,\land,\lor,$ Variablensymbolen ) definiert eine \textbf{boolsche Funktion} \\ $f_{A}: \left\lbrace 0,1\right\rbrace ^{n} \rightarrow \left\lbrace 0,1\right\rbrace;\;\;A$ heißt \textbf{tautologisch/Tautologie} wenn $f_{A}(a_{1},\dotsc, a_{n})=1$ für alle $a_{1},\dotsc, a_{n} \in \left\lbrace 0,1\right\rbrace$  
\item schreibe $A\Rightarrow B$ für $\lnot A\lor B, A\Leftrightarrow B$ für $(A\Rightarrow B)\land (B\Rightarrow A)$ (d.h. $\Rightarrow$ ist immer war außer für $1 \Rightarrow 0,\\ \Leftrightarrow$ ist wahr für $1\Leftrightarrow 1$ und $0\Leftrightarrow 0$); es gilt $A\Rightarrow B$ äquivalent zu $\lnot B\Rightarrow \lnot A$ (\textbf{Kontraposition})
\item Ausdrücke $A,B$ sind \textbf{äquivalent} wenn $f_{A}=f_{B}$ (d.h. wenn $A\Leftrightarrow B $ Tautologie); $A$ \textbf{impliziert} $B$ ($A\models B$) wenn $A\Rightarrow B$ Tautologie
\item \textbf{Darstellungssatz:} für jede $n$-stellige boolsche Funktion $f$ existiert Ausdruck $A$ in $n$ Variablen, so dass $f_{A}=f$
\item  \textbf{disjunktive Normalform} (DNF): $\bigvee_{i} \bigwedge_{j} L_{ij};\;\;$\textbf{konjunktive Normalform} (KNF): $\bigwedge_{i} \bigvee_{j} L_{ij}$\\
Beispiel: Sei $f$ gegeben durch Tabelle. Es gilt $f=f_{A}=f_{B}$.
\end{itemize}
 \begin{tabular}{c|c}
$a_{1},a_{2}$ & $f(a_{1},a_{2})$\\
\hline
0,0&0\\
0,1&1\\
1,0&0\\
1,1&1
\end{tabular} $A=\underbrace{(\underbrace{\lnot X_{1}}_{\text{\textbf{Literal} } L_{ij}} \land X_{2})\lor \underbrace{(X_{1}\land X_{2})}_{\text{\textbf{Klausel}}}}_{\text{DNF}}\;\;\;B=\underbrace{(X_{1}\lor X_{2})\land (\lnot X_{1}\lor X_{2})}_{\text{KNF}}$
\begin{itemize}
\item eine Aussage in \textit{KNF (!!)} heißt \textbf{Horn}, falls jede Klausel maximal ein positives Literal ($L_{ij}$ der Form $X$) enthält
\item Algorithmus für \textbf{Horn-SAT:} 
\begin{enumerate}
\item suche nach Klausel der Gestalt $X$, lösche dann alle Literale der Gestalt $\lnot X$
\item falls dadurch leere Klausel entsteht \texttt{return} NEIN
\item gehe zu 1. solange noch etwas gelöscht werden kann; danach \texttt{return} JA 
\end{enumerate}
Beispiel: $A=(X_{1}\lor \lnot X_{2}\lor \lnot X_{3})\land X_{3}\;\;\;$ Ausdruck ist Horn $\Rightarrow$ Horn-SAT Algorithmus anwendbar\\
$X_{3}$ ist eine Klausel $\Rightarrow \lnot X_{3}$ streichen \\
$A\Leftrightarrow (X_{1}\lor \lnot X_{2})\land X_{3}\;\;\;\;\; \Rightarrow$ JA (ist erfüllbar)
\item Beweise in Aussagenlogik über Ausdrücke umformen oder Wertetabelle
\end{itemize}


\subsection{Relationen}
%beachte hier evtl auch Grundlagen aus VL 1 und VL3
\end{document}