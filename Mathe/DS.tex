%Grundsätzliches: Enthalten sind die Gebieten die Frau Henke schon als Schwerpunkte angekündigt hat (und natürliche das nicht ausdrücklich geforderte Grundwissen auf dem sie aufbauen).
%Das meiste ist eher knapp gehalten, extrem grundlegende Dinge sind nicht nochmal erklärt weil ich denke dass das beim lernen eher stört.
%Selbstverständlich kannst du auch noch kürzen/ergänzen 

\documentclass[10pt,a4paper]{article}
\usepackage[utf8]{inputenc}
\usepackage{amsmath}
\usepackage{amsfonts}
\usepackage{amssymb}
\usepackage{amsthm}
\usepackage{graphicx}
\usepackage[margin=0.6in]{geometry}
\setcounter{section}{1}
\begin{document}



\section*{\textit{Prüfungsrelevante} Verfahren, Sätze und Rechenregeln}
\section{Diskrete Strukturen}



\subsection{Mengenlehre}



\subsection{Abbildungen}



\subsection{Beweis mittels vollständiger Induktion (Beispiel)}
\textit{Beweis.}
Die Aussage $A_{n}$ sei $\sum_{k=0}^{n} q^{k}= \dfrac{1-q^{n+1}}{1-q}$ mit $n \in \mathbb{N}, q \in \mathbb{R}, q \neq 1$.\\\\
(IA): $n_{0}=0$: $\sum_{k=0}^{0} q^{k}=q^{0}=1= \dfrac{1-q^{0+1}}{1-q}$ \hspace{0.3cm} w.A.\\
\hspace*{1cm}$\Rightarrow$ Es gilt $A_{0}$\\\\
(IV): $\forall \tilde{n}: n_{0} \leq \tilde{n} \leq n:\sum_{k=0}^{\tilde{n}} q^{k}= \dfrac{1-q^{\tilde{n}+1}}{1-q}$\\\\
(IS):$\sum_{k=0}^{n+1} q^{k}= \sum_{k=0}^{n} q^{k}+q^{n+1}\stackrel{\text{(IV)}}{=}\dfrac{1-q^{n+1}}{1-q}+q^{n+1}=\dfrac{1-q^{n+1}+(1-q)q^{n+1}}{1-q}$\\
\hspace*{2.23cm}$=\dfrac{1-q^{n+1}+q^{n+1}-q^{n+2}}{1-q}=\dfrac{1-q^{(n+1)+1}}{1-q}$\\\\
\hspace*{1cm}$\Rightarrow$ Damit ist die Behauptung für alle $n \in \mathbb{N}$ vollständig bewiesen
\qed\\\\
\begin{itemize}
\item "Die Aussage $A_{n}$ sei..." nur in VL und AuD Skript, \textit{evtl.} wird sonst aber z.Z.: erwartet; IV muss auch nicht unbedingt notiert werden
\item alles nochmal mit (n+1)+1 hinschreiben ist nicht nötig
\item Varianten: $A_{n} \Rightarrow A_{n+1}$/ aus $A_{n}$ folgt $A_{n+1}$ für alle $n\in \mathbb{N}$\\
 w.A. /Folglich gilt $A_{n}$ für alle $n \in \mathbb{N}, n \geq n_{0}$
\item Bedenke dass z.T. auch nur für $n \in \mathbb{N}, n\geq k$ bewiesen wird (kein $\tilde{n}$)!! und $n_{0}=0$ nicht immer gelten muss
\end{itemize}








\subsection{Zahlentheorie}



\subsection{Gruppentheorie}



\subsection{Graphentheorie}




\subsection{Aussagenlogik}



\subsection{Relationen}
\end{document}