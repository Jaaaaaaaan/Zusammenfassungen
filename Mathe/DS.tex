\documentclass[10pt,a4paper]{article}
\usepackage[utf8]{inputenc}
\usepackage{amsmath}
\usepackage{amsfonts}
\usepackage{amssymb}
\usepackage{amsthm}
\usepackage{graphicx}
\usepackage[margin=0.6in]{geometry}
\setcounter{section}{1}
\begin{document}



\section*{\textit{Prüfungsrelevante} Verfahren, Sätze und Rechenregeln}
\section{Diskrete Strukturen}
\subsection{Mengenlehre und Kombinatorik}
\begin{itemize}
\item zwei Mengen $A$ und $B$ sind gleich wenn sie die selben Elemente haben, d.h. wenn $A \subseteq B \land B \subseteq A$
\item Beachte z.B. dass $\left\lbrace\left\lbrace 1,2\right\rbrace 7\right\rbrace \nsubseteq \mathbb{N}$
\item Schnitt und Vereinigung sind kommutativ, assoziativ, distributiv in beide Richtungen; für Beweise kann es nützlich sein sich die Definitionen dieser Operationen in Erinnerung zu rufen, $\overline{A\cup B}=\overline{A}\cap \overline{B}, \overline{A\cap B}=\overline{A}\cup \overline{B}$
\item $A\times B=\left\lbrace (a,b)\mid a\in A \land b\in B\right\rbrace$ heißt \textbf{kartesisches Produkt} oder \textbf{Produktmenge},  $\vert A\times B\vert =\vert A\vert \cdot \vert B\vert$
\item \textbf{Potenzmenge} $\mathcal{P}(A)$ ist die Menge aller (auch unechten) Teilmengen von $A$, $\vert \mathcal{P}(A)\vert =2^{\vert A\vert}$, es gilt stets $\emptyset \in \mathcal{P}(A)$
\item $\begin{pmatrix} n\\ k\end{pmatrix}=\dfrac{n!}{k!(n-k)!}$
\item \textbf{Handschlaglemma:} Anzahl der Teilnehmer einer Konferenz, die einer ungeraden Anzahl von Teilnehmern die Hand geben, ist immer gerade

%genauer siehe Graphentheorie, ich hab versucht da eine Form zu finden die man sich besser merken kann aber das scheint tatsächlich nur in der Graphentheorie eine größere Rolle zu spielen

\end{itemize}



\subsection{Abbildungen}



\subsection{Beweis mittels vollständiger Induktion (Beispiel)}
\textit{Beweis.}
Die Aussage $A_{n}$ sei $\sum_{k=0}^{n} q^{k}= \dfrac{1-q^{n+1}}{1-q}$ mit $n \in \mathbb{N}, q \in \mathbb{R}, q \neq 1$.\\\\
(IA): $n_{0}=0$: $\sum_{k=0}^{0} q^{k}=q^{0}=1= \dfrac{1-q^{0+1}}{1-q}$ \hspace{0.3cm} w.A.\\
\hspace*{1cm}$\Rightarrow$ Es gilt $A_{0}$\\\\
(IV): $\forall \tilde{n}: n_{0} \leq \tilde{n} \leq n:\sum_{k=0}^{\tilde{n}} q^{k}= \dfrac{1-q^{\tilde{n}+1}}{1-q}$\\\\
(IS):$\sum_{k=0}^{n+1} q^{k}= \sum_{k=0}^{n} q^{k}+q^{n+1}\stackrel{\text{(IV)}}{=}\dfrac{1-q^{n+1}}{1-q}+q^{n+1}=\dfrac{1-q^{n+1}+(1-q)q^{n+1}}{1-q}$\\
\hspace*{2.23cm}$=\dfrac{1-q^{n+1}+q^{n+1}-q^{n+2}}{1-q}=\dfrac{1-q^{(n+1)+1}}{1-q}$\\\\
\hspace*{1cm}$\Rightarrow$ Damit ist die Behauptung für alle $n \in \mathbb{N}$ vollständig bewiesen
\qed\\\\
\begin{itemize}
\item "Die Aussage $A_{n}$ sei..." nur in VL und AuD Skript, \textit{evtl.} wird sonst aber z.Z.: erwartet; IV muss auch nicht unbedingt notiert werden
\item alles nochmal mit (n+1)+1 hinschreiben ist nicht nötig
\item Varianten: $A_{n} \Rightarrow A_{n+1}$/ aus $A_{n}$ folgt $A_{n+1}$ für alle $n\in \mathbb{N}$\\
 w.A. /Folglich gilt $A_{n}$ für alle $n \in \mathbb{N}, n \geq n_{0}$
\item Bedenke dass z.T. auch nur für $n \in \mathbb{N}, n\geq k$ bewiesen wird (kein $\tilde{n}$)!! und $n_{0}=0$ nicht immer gelten muss
\end{itemize}








\subsection{Zahlentheorie}



\subsection{Gruppentheorie}



\subsection{Graphentheorie}




\subsection{Aussagenlogik}



\subsection{Relationen}
\end{document}